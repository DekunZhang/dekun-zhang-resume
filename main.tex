\documentclass{resume} % Utilizing the custom resume.cls style
\usepackage{multicol}
\usepackage[left=0.4 in,top=0.4in,right=0.4 in,bottom=0.4in]{geometry} % Document margins
\newcommand{\tab}[1]{\hspace{.2667\textwidth}\rlap{#1}} 
\newcommand{\itab}[1]{\hspace{0em}\rlap{#1}}
\name{Dekun Zhang} % Your name
\address{+44 7470092795 \\
\href{mailto:dekun.zhang.20@ucl.ac.uk}{dekun.zhang.20@ucl.ac.uk} \\ 
\href{https://www.linkedin.com/in/dekun-zhang/}{Linkedin} \\  %
\href{https://github.com/DekunZhang/}{GitHub}}

\begin{document}

\begin{rSection}{Education}

{\bf BSc Computer Science}, UCL \hfill {2021 - Present}\\
First Year: Earned Upper Second-Class Honours, averaging 66\% \\
Second Year: Achieved First-Class Honours, averaging 75\% \\
Relevant Coursework: Algorithm, Computer Architecture and Concurrency, Discrete Mathematics, \\
Engineering Mathematics in Finance, Logic, Object-Oriented Programming, Security, \\
Software Engineering, Computability and Complexity Theory, Compilers, Data Mining and Analysis, \\
Stochastic Calculus and Uncertainty Analysis, Computer Graphics, Machine Learning for Visual Computing

{\bf Undergraduate Preparatory Certificate}, UCL CLIE \hfill {2020 - 2021}\\
Modules: Mathematics (93\%), Physics (93\%), Science and Society (79\%), Academic English (72\%)\\
Final Grade: 84\%, Distinction

\end{rSection}

\begin{rSection}{SKILLS}

\begin{tabular}{ @{} >{\bfseries}l @{\hspace{6ex}} l }
Programming Languages & (Proficient) Python, Java, C; (Familiar) Haskell, HTML, C++, Shell
\\
English Proficiency & CEFR C1 Level (2021) Overall 7, Listening 6.5, Reading 8, Writing 7, Speaking 7
\\
Software Proficiency & Git, GitHub, \LaTeX, Arduino, Microsoft Office, Zotero, XMind, 3D Builder
\\
Hardware & 3D Printing, Basic Circuit Design and Implementation
\\
System Operation & Proficient with Vim, SSH, SCP, Nginx, Docker
\\
Network Management & Expertise in SD-WAN (Wireguard, ZeroTier, SoftEther, n2n), \\ 
& FRP(fast reverse proxy), VLAN, Switch and Router Operations (ufw, iptables)\\
Soft Skills & Proficient in Problem-solving, Critical thinking, Stress management, Adaptability
\end{tabular}
\end{rSection}

\begin{rSection}{WORK EXPERIENCE}

\textbf{Student Ambassador} \hfill Nov 2021 - Aug 2022\\
UCL Centre for Languages and International Education \hfill \textit{London, UK}
\begin{itemize}
    \itemsep -3pt {} 
    \item Assisted potential students in grasping the essence of UPC studies and led comprehensive campus tours.
\end{itemize}

\textbf{Transition Mentor} \hfill Sep 2021 - Dec 2021\\
UCL Centre for Languages and International Education \hfill \textit{London, UK}
\begin{itemize}
    \itemsep -3pt {} 
    \item Rendered advice, guidance, and support to newcomers during the initial term of their programme.
\end{itemize}
 
\textbf{IT Manager, Assistant General Manager} \hfill May 2021 - Sep 2021\\
Zhengda Dingsheng Financial Management Consulting Ltd, Gansu Branch. \hfill \textit{Lanzhou, China}
\begin{itemize}
    \itemsep -3pt {} 
    \item Managed procurement of hardware including personal computers, printers and routers during the company's inception, established intranet connectivity, and advertised job postings online.
    \item Assisted the General Manager in procuring office furniture and managing reimbursements.
    \item Troubleshot and resolved difficulties pertaining to office software usage for new employees.
\end{itemize}

\end{rSection} 

\begin{rSection}{RESEARCH EXPERIENCE}

\textbf{Summer Research Intern} \hfill Mar 2023 - Present\\
UCL Computer Science Department \hfill \textit{London, UK}\\
Participating in collaborative research with Dr. Sergey Mechtaev.

\textbf{Research Project} \hfill 2021\\
UCL Centre for Languages and International Education \hfill \textit{London, UK}\\
Supervisors: Leah Nicholls, Elizabeth Barnett
\begin{itemize}
    \itemsep -3pt {} 
    \item Investigated the potential of photovoltaic power generation as a primary energy source in China.
\end{itemize}

\end{rSection}
%----------------------------------------------------------------------------------------
% PROJECTS	
%----------------------------------------------------------------------------------------
\begin{rSection}{PROJECTS}
    \vspace{-1.25em}
    \item \textbf{Comprehensive IFRC Data Analysis System} {Engineered an intricate system to expedite the process of data procurement, sanitation, categorisation, consolidation, segmentation, and evaluation of historical, high-frequency, low-impact natural disasters from public data repositories, inclusive of national disaster loss databases (DesInventar and EM-DAT). This initiative's culmination was disseminated via the IFRC GO platform and made accessible as open-source on GitHub Repositories, thus, fostering further ingenuity and evolution in this arena. Collaborated with the International Federation of Red Cross and Red Crescent Societies.
    \href{https://students.cs.ucl.ac.uk/2022/group5/index.html}{(Accessible here)}}
    \item \textbf{ESP32-MQTT-WOL} {Undertook research and development of software harnessing Internet of Things (IoT) devices to facilitate remote activation of devices nestled within a local area network employing the MQTT protocol.}
    \item \textbf{Shell} {Architected a command-line interface application with the proficiency to execute a diverse range of commands, comprising cd, ls, cut, grep, and head, through the utilisation of Python programming language and ANTLR parser generator.}
    \item \textbf{tensorflow-conda-env} {Formulated an assembly of scripts to streamline the administration of conda environments featuring tensorflow GPU capabilities.}
    \item \textbf{ChatGPT Docstring Generator} {Curated a comprehensive guide designed to aid in the generation of high-calibre docstrings for provided code, leveraging the capabilities of ChatGPT.}
    \item \textbf{Next-gen music therapist for psychiatric patients} {Conceptualized and brought to fruition during the 2023 UCL CSS x TechSoc Hackathon, the project encapsulates a wearable device integrated with sensors and an AI-driven music therapy designer, a mobile application, and an administrative portal website. It further features a cloud-based therapist able to detect patients necessitating urgent medical attention.}
    \item \textbf{Bioreactor} {Crafted with heating, stirring, and pH subsystems, and interfaced with ThingsBoard via RPC protocol. Integrated the Arduino UNO and ESP32 using the I2C protocol.}
\end{rSection} 
%----------------------------------------------------------------------------------------
\begin{rSection}{Accolades} 
\begin{itemize}
    \item Third prize laureate in the Shaanxi Olympiad in Informatics \hfill 2020
    \item Distinguished as second-place winner and first-prize recipient in \\
    the Shaanxi Adolescent Robotics Competition. \hfill 2019
\end{itemize}
\end{rSection}
%----------------------------------------------------------------------------------------
\begin{rSection}{Extracurricular Endeavors} 
\begin{itemize}
    \item 	Participated in the upkeep of OpenWrt (a customisable operating system for routers) and Xray (a network proxy).
    \item	Instigated a personal blog and regularly updated with nine entries, each tackling intricate technical conundrums.
    \item   Conceived a five-part video tutorial series, amassing a total of 250 subscribers, 43,889 views, and 293 likes.
    \item   Volunteered as an audio editor and post-processor for a gaming news channel, amassing a substantial following of 72,203 subscribers, with over 2,097,000 plays.
\end{itemize}
\end{rSection}

%----------------------------------------------------------------------------------------
\begin{rSection}{Leadership} 
\begin{itemize}
    \item Enacted as the President of the Robotics Society in high school. Through my guidance and strategic direction, society members triumphed in the Shaanxi Adolescent Robotics Competition, securing all three top positions.
\end{itemize}
\end{rSection}

\begin{rSection}{Reference} 
    Available upon request.
\end{rSection}

\end{document}

